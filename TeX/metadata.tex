% ------------------------------------------------------------------------------
% PROMĚNNÉ PROSTŘEDÍ
% ------------------------------------------------------------------------------

% typ práce
\newcommand{\typPrace}{Absolventská práce}

% jméno autora práce
\newcommand{\autor}{Josef Novák}

% prohlášení - tvar vypracoval/vypracovala
\newcommand{\osoba}{vypracoval/a}

% studijní obor a program
\newcommand{\studijniObor}{Hudba/Zpěv}
% pro obor zpěv ponechte program prázdný (~)
\newcommand{\studijniProgram}{Hra na~nástroj}

% téma absolventské písemné práce
\newcommand{\appTema}{Název\\ absolventské práce}

% vedoucí práce a vyučující hlavního oboru
\newcommand{\appVedouci}{Mgr.~Jiří Vedoucí}
\newcommand{\appUcitel}{Mgr.\,MgA.~Pavel Hráč}

% škola a místo
\newcommand{\skola}{Konzervatoř Pardubice}
\newcommand{\misto}{Pardubice}

% rok odevzdání/obhajoby APP
% - pro aktuální kalendářní rok: \the\year{}
\newcommand{\skolniRok}{\the\year{}}

% datum odevzdání
% - pro datum překladu do PDF: \today
% - při vlastním datu zapište ve formátu "27. dubna 2022"
\newcommand{\datumOdevzdani}{\today}

% slovíčko "Shrnutí" v cizím jazyce
\newcommand{\shrnutiCizi}{Summary} % Angličtina
%\newcommand{\shrnutiCizi}{Zusammenfassung} % Němčina

% slovíčko "Klíčová slova" v cizím jazyce
\newcommand{\klicovaSlovaCiziLit}{Keywords} % Angličtina
%\newcommand{\klicovaSlovaCiziLit}{Schlüsselwörter} % Němčina

% čárkami oddělený seznam klíčových slov
% klíčová slova CZ
\newcommand{\klicovaslova}{slovo1, slovo2, slovo3, slovo4, slovo5}
% klíčová slova v cizím jazyce
\newcommand{\klicovaslovaCizi}{word1, word2, word3, word4, word5}

% pokud jsou použity pevné desky, vložit prázdnou stránku
% před titulní stranu a prázdnou stránku na konec dokumentu
%\def\isCover{1}
