\nadpis{První nadpis}

\noindent
Nadpisy první úrovně automaticky začínají na nové straně.
Použití citace.~\cite{harry}

\podnadpis{První podnadpis}
\noindent Text.\footnote{Poznámka pod čarou.} Nota \nota{Fb}


\begin{figure}[!ht]
	\begin{center}
		\includegraphics[width=60mm]{./obrazky/kp-logo.pdf}
    \end{center}

	\caption{Logo: Konzervatoř Pardubice}
\end{figure}

\begin{figure}[!ht]
	\begin{center}
		\includegraphics[width=30mm]{./obrazky/kp-logo.pdf}
    \end{center}

	\caption{Mini logo: Konzervatoř Pardubice}
\end{figure}

\podpodnadpis{První podpodnadpis}
\noindent Ukázka.

\nadpis{Druhá kapitola}
\noindent Text druhé kapitoly.

\podnadpis{podnadpis 2.1}

\podpodnadpis{podpodnadpis 2.1.1}

\podnadpis{podnadpis 2.2}

\podnadpis{podnadpis 2.3}

\podpodnadpis{podpodnadpis 2.3.1}
\podpodnadpis{podpodnadpis 2.3.2}
\podpodnadpis{podpodnadpis 2.3.3}
