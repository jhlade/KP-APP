\nadpis{Šablona a~formální požadavky APP}
Kompletní požadavky k~absolventským písemným pracím shrnuje směrnice
Konzervatoře Pardubice č.~5/19, jejíž plné znění je k~dispozici na webových
stránkách školy v~nabídce \textit{Studium – Absolutorium}.
 
Šablona si klade za cíl zjednodušit formální stránku práce, aby měli autoři
více času na samotný obsah.


Nadpisy první úrovně automaticky začínají na nové straně.
Použití citace.~\cite{harry}


\podnadpis{Doporučená minimální struktura práce}
Základní struktura absolventské práce udává zejména závazné pořadí nečíslovaných
oddílů a kapitol, které se nacházejí mimo vlastní text práce. Mezi základní
požadované prvky patří \textit{Titulní strana}, \textit{Prohlášení
o~samostatnosti}, \textit{Obsah}, \textit{Úvod}, vlastní text práce,
\textit{Závěr}, \textit{Shrnutí}, \textit{Shrnutí v~cizím jazyce},
\textit{Seznam literatury} a~přílohy.\footnote{Všechny zmíněné části jsou
v~předkládaných šablonách již předpřipraveny}

\podnadpis{Rozsah práce}
Minimálním rozsahem práce se rozumí nejnižší možný počet stran vlastního textu,
obvykle od~\textit{Úvodu} po konec \textit{Závěru}, udávaný v~normostranách.
Pojem normostrana je definován jako strojopis 30 řádků po 60 úhozech, tedy
přesně 1\,800 znaků včetně mezer. Šablony jsou koncipovány tak, aby v~nich psaná
jedna strana A4 odpovídala přibližně 1\,400~–~2\,400 úhozům, což je přibližně
v~rozsahu ±30~\% normostrany.

Do rozsahu vlastního textu se dále také počítají textové přílohy, ale pouze
jsou-li autorským dílem (např.~provedené rozhovory). V~této ukázce je jako
příklad první přílohy přepis korespondence, který autorským dílem \textit{není}.

Pro APP je minimální hranicí 20 stran textu, samozřejmě je lépe, je-li stran
napsáno více, optimum bude cca na 22-25 tiskových stranách. Horní hranici je
nutné konzultovat s~vedoucím práce, rozsah by však neměl být zbytečně objemný.

\podnadpis{Formální úprava}
Požadavky na formální úpravu zahrnují především vzhled stránky, vše je již
obsaženo v této šabloně. Zejména se jedná o~formát papíru A4, okraje stránky
po 2,5~cm shora, zespodu a~zprava, levý okraj 4~cm.  Formální úprava dále
specifikuje velikost písma a~proklad řádkování vlastního textu, způsob číslování
a~umístění kapitol, stránek a~příloh.

\podnadpis{Forma odevzdání}
Práce se odevzdává výhradně ve~formátu PDF (Portable Document Format). PDF
soubor \texttt{app.pdf} vznikne automaticky překladem v~prostředí \XeLaTeX
jednoduchým zavoláním \texttt{make}.

% ------------------------------------------------------------------------------
\nadpis{Nadpis první úrovně}
Číslovaný nadpis první úrovně lze realizovat příkazem
\texttt{{\textbackslash}nadpis\{Text nadpisu\}}. První úroveň nadpisu otevírá
každou hlavní kapitolu práce se~samostatným tematickým celkem a~vždy začíná
na~nové stránce. Nadpis hlavní kapitoly nikdy nemůže okamžitě navazovat na~dílčí
podnadpisy, musí za~ním stát krátký odstavec stručně charakterizující další
obsah.

\podnadpis{Nadpis druhé úrovně}
Číslovaný nadpis druhé úrovně je k~dispozici pod~příkazem
\texttt{{\textbackslash}podnadpis\{Text podnadpisu\}}. Druhá úroveň člení každou
kapitolu na~její stěžejní body a~obvykle se zde nachází nejvíce odstavců textu.

\podpodnadpis{Nadpis třetí úrovně}
Nadpis třetí úrovně je uveden příkazem
\texttt{{\textbackslash}podpodnadpis\{Text podpodnadpisu\}}. Závěrečné práce by
neměly přesáhnout tři úrovně členění textu a~třetí úroveň je obecně vhodné
věnovat pro~popisy konkrétních případů, které obecně zahrnuje nadpis druhé
úrovně (př.~\textit{1 - Hudba}, \textit{2 - Hudební nástroje}, \textit{3 -
Smyčcové nástroje}).

\podnadpis{Jiné objekty}
Harry Potter nikdy nevydal tón \nota{Eb2}, natož \nota{C#6}, přestože o~tom
v~knížce~\cite{harry} není ani slovo. Následující obrázek
zachycuje logo školy.

\begin{figure}[!ht]
	\begin{center}
		\includegraphics[width=60mm]{./obrazky/kp-logo.pdf}
    \end{center}

	\caption{Logo: Konzervatoř Pardubice}
\end{figure}

\noindent
Po~obrázku, nebo na nové straně, je třeba manuálně zrušit automatické odsazování
prvního řádku. \lipsum[5-7][12-22]

\podnadpis{Podnadpis 2.3}
\lipsum[5-7][12-22]

\podpodnadpis{Podpodnadpis 2.3.1}
\lipsum[5-7][12-22]

\lipsum[5-7][12-22]

\podpodnadpis{Podpodnadpis 2.3.2}
\lipsum[5-7][12-22]

\lipsum[5-7][12-22]

\podpodnadpis{Podpodnadpis 2.3.3}
\lipsum[5-7][12-22]

\lipsum[5-7][12-22]