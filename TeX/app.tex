%-------------------------------------------------------------------------------
% ABSOLVENTSKÁ PÍSEMNÁ PRÁCE
%-------------------------------------------------------------------------------

\documentclass[12pt,a4paper,oneside]{report}

%-------------------------------------------------------------------------------
% KONFIGURACE PRÁCE
%-------------------------------------------------------------------------------

% základní konfigurace a balíčky
%%%%%%%%%%%%%%%%%%%%%%%%%%%%%%%%%%%%%%%%%%%%%%%%%%%%%%%%%%%%%%%%%%%%%%%%%%%%%%%%

% metadata práce
%%%%%%%%%%%%%%%%%%%%%%%%%%%%%%%%%%%%%%%%%%%%%%%%%%%%%%%%%%%%%%%%%%%%%%%%%%%%%%%%
% PROMĚNNÉ PROSTŘEDÍ
%%%%%%%%%%%%%%%%%%%%%%%%%%%%%%%%%%%%%%%%%%%%%%%%%%%%%%%%%%%%%%%%%%%%%%%%%%%%%%%%

% jméno autora práce
\newcommand{\autor}{Josef Novák}

% studijní obor a program
\newcommand{\studijniObor}{Hudba}
\newcommand{\studijniProgram}{Hra na nástroj}

% téma absolventské písemné práce
\newcommand{\appTema}{Název absolventské práce}

% vedoucí práce a vyučující hlavního oboru
\newcommand{\appVedouci}{Mgr. Jiří Vedoucí}
\newcommand{\appUcitel}{MgA. Pavel Hráč}

% škola a místo
\newcommand{\skola}{Konzervatoř Pardubice}
\newcommand{\misto}{Pardubice}

% prohlášení - tvar vypracoval/vypracovala
\newcommand{\osoba}{vypracoval}

% rok odevzdání/obhajoby APP
\newcommand{\skolniRok}{\the\year{}}

% datum odevzdání - pro datum překladu do PDF \today
\newcommand{\datumOdevzdani}{\today}

% shrnutí v cizím jazyce
\newcommand{\shrnutiCizi}{Summary} % Angličtina
%\newcommand{\shrnutiCizi}{Zusammenfassung} % Němčina

% "klíčová slova" v cizím jazyce
\newcommand{\klicovaSlovaCiziLit}{Keywords} % Angličtina
%\newcommand{\klicovaSlovaCiziLit}{Schlüsselwörter} % Němčina

% klíčová slova CZ
\newcommand{\klicovaslova}{slovo1, slovo2, slovo3, slovo4, slovo5}
% klíčová slova v cizím jazyce
\newcommand{\klicovaslovaCizi}{word1, word2, word3, word4, word5}


%%%%%%%%%%%%%%%%%%%%%%%%%%%%%%%%%%%%%%%%%%%%%%%%%%%%%%%%%%%%%%%%%%%%%%%%%%%%%%%%

\renewcommand{\baselinestretch}{1.2}

\usepackage{pdfpages} % vnoření PDF
\usepackage{newclude}

\usepackage[main=czech]{babel} % čeština
\usepackage{csquotes} % české uvozovky

%%%%%%%%%%%%%%%%%%%%%%%%%%%%%%%%%%%%%%%%%%%%%%%%%%%%%%%%%%%%%%%%%%%%%%%%%%%%%%%%
% LITERATURA A REFERENCE
%%%%%%%%%%%%%%%%%%%%%%%%%%%%%%%%%%%%%%%%%%%%%%%%%%%%%%%%%%%%%%%%%%%%%%%%%%%%%%%%

% literatura - ISO 690
\usepackage[
   backend=biber
  ,style=iso-numeric
  ,sortlocale=cs_CZ
  ,sorting=nyt
  ,maxnames=3
  ,minnames=3
  ,pagetotal=true
  ,autolang=other
  ,bibencoding=UTF8
  ,spacecolon=false % mezera před dvojtečkou
]{biblatex}

\setcounter{biburllcpenalty}{7000}
\setcounter{biburlucpenalty}{8000}

\addbibresource{./literatura.bib}

% formátování seznamu zdrojů a literatury
\DeclareFieldFormat{labelnumberwidth}{[\,{#1}\,]}
\defbibenvironment{bibliography}
  {\list
     {\printtext[labelnumberwidth]{%
        %%\printfield{prefixnumber}%
        \printfield{labelprefix}%
        \printfield{labelnumber}}}
     {\setlength{\labelwidth}{\labelnumberwidth}%
      \setlength{\leftmargin}{\labelwidth}%
      \setlength{\labelsep}{\biblabelsep}%
      \addtolength{\leftmargin}{+1.5\labelsep}%
      \setlength{\itemsep}{\bibitemsep}%
      \setlength{\parsep}{\bibparsep}}%
      \renewcommand*{\makelabel}[1]{\hss##1}}
  {\endlist}
  {\item}

%%%%%%%%%%%%%%%%%%%%%%%%%%%%%%%%%%%%%%%%%%%%%%%%%%%%%%%%%%%%%%%%%%%%%%%%%%%%%%%%

% přílohy
\usepackage[titletoc]{appendix}

% tabulky a poznámky pod čarou
\usepackage{tabularx}
\usepackage{makecell}
\usepackage{footnote}

% obrázky v textu
\usepackage{wrapfig}
\usepackage{subcaption}

% matematické prostředí
\usepackage{amsmath}
\usepackage{mathtools}

% stojatá řecká písmena
\usepackage{upgreek}

% rámečky
\usepackage{framed}

% barvy
\usepackage{xcolor}

% formátování a záhlaví
\usepackage{xunicode,xltxtra,url,parskip}
\setlength{\parindent}{1cm}
\usepackage{fancyhdr}

% kreslení v TikZ
\usepackage{tikz,pgfplots}
\pgfplotsset{compat=1.18}
\usepackage[simplified]{pgf-umlcd}
\usepgfplotslibrary{polar}
\tikzset{fontscale/.style = {font=\relscale{#1}}}
\usepackage[siunitx,european]{circuitikz} % evropské značky, jednotky SI
\usetikzlibrary{intersections}

% úprava nadpisů a řádkování
\usepackage{titlesec}
\usepackage{setspace}

% změna záhlaví, hypertextové odkazy bez rámečků
\usepackage{chngcntr}
\usepackage[hidelinks]{hyperref}
\hypersetup{
	colorlinks=false,
	pdfborder={0 0 0},
}

% ovládání software, klávesnice
\usepackage{menukeys}

% vložené zdrojové kódy
\usepackage{verbatimbox}

\makeatletter
\newcommand\verbfilelist[2][]{%
  \setcounter{VerbboxLineNo}{0}%
  \def\verbatim@processline{%
    {\addtocounter{VerbboxLineNo}{1}%
    #1\setbox0=\hbox{#1\the\verbatim@line}%
    \hsize=\wd0 \the\verbatim@line\par}}%
  \verbatiminput{#2}
  \let\verbatim@processline\sv@verbatim@processline
}
\makeatother

%%% STRÁNKA A KAPITOLY %%%%%%%%%%%%%%%%%%%%%%%%%%%%%%%%%%%%%%%%%%%%%%%%%%%%%%%%%

% okraje stránky
\usepackage[
  top=25mm,
  bottom=25mm,
  left=40mm,
  right=25mm,
  %includehead,
  includefoot,
  heightrounded, % underfull zarovnání
]{geometry}

%\renewcommand*\thechapter{\arabic{chapter}}

% číslování obrázků
\counterwithin{figure}{chapter}

% čísla rovnic pod kapitolou a pod podkapitolou
\counterwithin*{equation}{chapter}
\counterwithin*{equation}{section}

% čísla tabulek
\counterwithin{table}{section}

% nadpis kapitoly
\titleformat{\chapter}{\bf\raggedright\Large}{\thechapter.~}{0em}{}
\titlespacing{\chapter}{0pt}{3pt}{3pt}

% podnadpis
%\titleformat{\section}{\bf\raggedright\large}{\thechapter.\thesection.~}{0em}{}
\titlespacing{\section}{0pt}{3pt}{3pt}

% podpodnadpis
%\titleformat{\subsection}{\bf\raggedright}{\thechapter.\thesubsection.~}{0em}{}

% podpodnadpis
%\titleformat{\subsubsection}{\bf\raggedright}{\thesubsubsection.~}{0em}{}
%\titlespacing{\subsubsection}{0pt}{3pt}{3pt}

% příkazy
\newcommand{\nadpis}[1]{
    \chapter{#1}
}

\newcommand{\podnadpis}[1]{
    \section{#1}
}

\newcommand{\podpodnadpis}[1]{
    \subsection{#1}
}

% kapitola vždy začíná na nové stránce
%\AddToHook{cmd/section/before}{\clearpage}

% výchozí záhlaví a zápatí stránky
\lhead{}
\rhead{}
\cfoot{}

% základní styl
\fancypagestyle{StyleBase} {
	\fancyhf{}
	\lhead{}
	\rhead{}
	\cfoot{\thepage}
	\rfoot{}
}

% styl příloh
\fancypagestyle{StyleAppendix} {
	\fancyhf{}
	\lhead{}
	\rhead{}
	\cfoot{\thepage}
	\rfoot{Přílohy}
}

% prázdná stránka
\fancypagestyle{StyleBlank} {
	\fancyhf{}
	\lhead{}
	\rhead{}
	\cfoot{}
	\rfoot{}
	\renewcommand{\headrulewidth}{0pt}
}

%%% VLASTNÍ PŘÍKAZY %%%%%%%%%%%%%%%%%%%%%%%%%%%%%%%%%%%%%%%%%%%%%%%%%%%%%%%%%%%%

% tečkovaná vodící čára pro podpisy
\newcommand{\dotrule}[1]{%
   \parbox[t]{#1}{\dotfill}}

% nadpis bez čísel
\newcommand{\nadpisBezCisla}[1]{
	\newpage
    \setcounter{chapter}{0}
    \setcounter{section}{0}
    \chapter*{#1}
    \addcontentsline{toc}{chapter}{#1}
}

% dvojité podtržení
\def\doubleunderline#1{\underline{\underline{#1}}}

% nota - tón
\newcommand\note[1]{\xnote#1\relax\relax\relax}
	\def\xnote#1#2#3\relax{#1\if#2\relax\else\if b#2$\flat\if#3\relax%
	\else_{#3}\fi$\else\if###2$\sharp\if#3\relax\else_{#3}\fi$\else$_{#2}$\fi\fi\fi}

% záporná nota
\def\noteminus{%
  \setbox0=\hbox{-}%
  \vcenter{%
    \hrule width\wd0 height \the\fontdimen8\textfont3%
  }%
}

% volná, nečíslovaná stránka
\newcommand\blankpage{%
    \null
    \clearpage
    \thispagestyle{empty}%
    %\addtocounter{page}{-1}%
    \newpage
    \clearpage}

\newlength\longest

%%%%%%%%%%%%%%%%%%%%%%%%%%%%%%%%%%%%%%%%%%%%%%%%%%%%%%%%%%%%%%%%%%%%%%%%%%%%%%%%

% slovník
% slovník pro korektní zalamování slov

% en
\hyphenation{Lily-Pond}
\hyphenation{Muse-Score}


% PDF metadata
\author{\autor}
\hypersetup
{
    pdfauthor={\autor},
    pdfsubject={Absolventská písemná práce {\skolniRok}},
    pdftitle={\appTema}
}



%-------------------------------------------------------------------------------
% ZAČÁTEK DOKUMENTU
%-------------------------------------------------------------------------------

\begin{document}

%-------------------------------------------------------------------------------
% TITULNÍ STRÁNKA
%-------------------------------------------------------------------------------

% titulní list

\pagestyle{empty}

\begin{center}

	\begin{Large}
		\noindent
		{\textbf{\MakeUppercase{\skola}}}
	\end{Large}

	\vfill

	\begin{Large}
		\noindent
		{\typPrace}
	\end{Large}

	\begin{spacing}{1.8}
		\begin{huge}
			\noindent
			\textbf{\MakeUppercase{\appTema}}
		\end{huge}
	\end{spacing}

	\vfill

	\begin{Large}
		\noindent
		\textbf{\autor}
	\end{Large}


	\begin{large}
		\noindent
		Obor: {\studijniObor}

		\noindent
		{\studijniProgram}
	\end{large}

\end{center}

\vfill

\begin{large}
	\begin{tabularx}{\textwidth}{ l X }
		Vedoucí práce:	& {\appVedouci} \\
		Pedagog hlavního oboru:	& {\appUcitel} \\
	\end{tabularx}
\end{large}

\vfill

\begin{center}
	\begin{large}
		\noindent
		\textbf{\MakeUppercase{\misto}}

		\noindent
		\textbf{\MakeUppercase{\skolniRok}}
	\end{large}
\end{center}

\newpage


%-------------------------------------------------------------------------------
% PROHLÁŠENÍ
%-------------------------------------------------------------------------------

% prohlášení o samostatnosti

\pagestyle{empty}

{~}
\vfill

\noindent
Čestně prohlašuji, že jsem absolventskou práci {\osoba} samostatně \\
s~použitím uvedených pramenů a~literatury.

\vfill

\begin{spacing}{1.2}
\noindent
\begin{tabularx}{\textwidth}{ X c }
   V~Pardubicích     & \dotrule{3.5cm} \\
   {\datumOdevzdani} & {\autor} \\
\end{tabularx}
\end{spacing}

\newpage


%-------------------------------------------------------------------------------
% PODĚKOVÁNÍ
%-------------------------------------------------------------------------------

% poděkování

\pagestyle{empty}

{~}
\vfill

\noindent
List s~textem poděkování není povinný a~lze jej vynechat zakomentováním vstupu
v~souboru \texttt{app.tex}.

\newpage


%-------------------------------------------------------------------------------
% OBSAH A SEZNAM PŘÍLOH
%-------------------------------------------------------------------------------

\pagestyle{StyleBase}
\setcounter{tocdepth}{3}

% obsah
\addtocontents{toc}{\protect\thispagestyle{StyleBlank}}
\tableofcontents

% seznam příloh

%-------------------------------------------------------------------------------
% VLASTNÍ TEXT PRÁCE
%-------------------------------------------------------------------------------

% Úvod -------------------------------------------------------------------------
\nadpisBezCisla{Úvod}
\setcounter{page}{1}

\noindent
Úvod předkládá metodologickou a~koncepční charakteristiku práce.

% Práce ------------------------------------------------------------------------

\nadpis{První nadpis}

\noindent
Nadpisy první úrovně automaticky začínají na nové straně.
Použití citace.~\cite{harry}

\podnadpis{První podnadpis}
\noindent Text.\footnote{Poznámka pod čarou.}

\podpodnadpis{První podpodnadpis}
\noindent Ukázka.

\nadpis{Druhá kapitola}
\noindent Text druhé kapitoly.

\podnadpis{podnadpis 2.1}

\podpodnadpis{podpodnadpis 2.1.1}

\podnadpis{podnadpis 2.2}

\podnadpis{podnadpis 2.3}

\podpodnadpis{podpodnadpis 2.3.1}
\podpodnadpis{podpodnadpis 2.3.2}
\podpodnadpis{podpodnadpis 2.3.3}

% Závěr ------------------------------------------------------------------------

\nadpisBezCisla{Závěr}

\noindent
Závěr dává odpověď cíle, otázky a~úkoly stanovené v~Úvodu práce.

%-------------------------------------------------------------------------------
% SHRNUTÍ
%-------------------------------------------------------------------------------

\nadpisBezCisla{Shrnutí}
\noindent
Shrnutí je stručným shrnutím problematiky práce. Je tvořeno souvislými,
samostatnými větami a~obvykle nepřesahuje deset řádků textu. Strukturu vět je
vhodné rozložit podle jednotlivých hlavních kapitol. Př.:~Následující práce
pojednává o~specifických problémech psaní absolventských prací v~prostředí
Konzervatoře Pardubice. Nejprve jsou uvedeny nejčastěji používané postupy při
psaní obecných závěrečných prací a~je také definováno celkové prostředí
Konzervatoře se zvláštním zaměřením na její jednotlivá oddělení. Praktickou
částí práce je komparativní analýza vybraných postupů aplikovaných na jednotlivá
oddělení. Nedílnou součástí je dále vyhodnocení získaných poznatků a~závěrečná
doporučení.

\noindent
\textbf{Klíčová slova: } \klicovaslova


%-------------------------------------------------------------------------------
% SHRNUTÍ V CIZÍM JAZYCE
%-------------------------------------------------------------------------------

\nadpisBezCisla{\shrnutiCizi}
Summary in foreign language.

\noindent
\textbf{\klicovaSlovaCiziLit: } \klicovaslovaCizi


%-------------------------------------------------------------------------------
% LITERATURA A ZDROJE
%-------------------------------------------------------------------------------

\cleardoublepage\phantomsection
\addcontentsline{toc}{chapter}{Literatura a~dalšízdroje}

% Literatura a další zdroje
\begingroup
	\printbibliography[title={Literatura a~další zdroje}]
\endgroup

\phantomsection

\pagestyle{empty}
\cleardoublepage

%-------------------------------------------------------------------------------
% PŘÍLOHY
%-------------------------------------------------------------------------------

%-------------------------------------------------------------------------------
% KONEC DOKUMENTU
%-------------------------------------------------------------------------------

\end{document}
