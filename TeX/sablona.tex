%%%%%%%%%%%%%%%%%%%%%%%%%%%%%%%%%%%%%%%%%%%%%%%%%%%%%%%%%%%%%%%%%%%%%%%%%%%%%%%%

% metadata práce
% ------------------------------------------------------------------------------
% PROMĚNNÉ PROSTŘEDÍ
% ------------------------------------------------------------------------------

% jméno autora práce
\newcommand{\autor}{Josef Novák}

% prohlášení - tvar vypracoval/vypracovala
\newcommand{\osoba}{vypracoval/a}

% studijní obor a program
\newcommand{\studijniObor}{Hudba/Zpěv}
% pro obor zpěv ponechte program prázdný (~)
\newcommand{\studijniProgram}{Hra na nástroj}

% téma absolventské písemné práce
\newcommand{\appTema}{Název\\ absolventské práce}

% vedoucí práce a vyučující hlavního oboru
\newcommand{\appVedouci}{Mgr.~Jiří Vedoucí}
\newcommand{\appUcitel}{MgA.~Pavel Hráč}

% škola a místo
\newcommand{\skola}{Konzervatoř Pardubice}
\newcommand{\misto}{Pardubice}

% rok odevzdání/obhajoby APP
\newcommand{\skolniRok}{\the\year{}}

% datum odevzdání - pro datum překladu do PDF \today
\newcommand{\datumOdevzdani}{\today}

% "Shrnutí" v cizím jazyce
\newcommand{\shrnutiCizi}{Summary} % Angličtina
%\newcommand{\shrnutiCizi}{Zusammenfassung} % Němčina

% "Klíčová slova" v cizím jazyce
\newcommand{\klicovaSlovaCiziLit}{Keywords} % Angličtina
%\newcommand{\klicovaSlovaCiziLit}{Schlüsselwörter} % Němčina

% klíčová slova CZ
\newcommand{\klicovaslova}{slovo1, slovo2, slovo3, slovo4, slovo5}
% klíčová slova v cizím jazyce
\newcommand{\klicovaslovaCizi}{word1, word2, word3, word4, word5}


% formát A4, jednostranně, základní velikost písma 12 bodů
\documentclass[12pt,a4paper,oneside]{report}

%%%%%%%%%%%%%%%%%%%%%%%%%%%%%%%%%%%%%%%%%%%%%%%%%%%%%%%%%%%%%%%%%%%%%%%%%%%%%%%%

\renewcommand{\baselinestretch}{1.2}

\usepackage{pdfpages} % vnoření PDF
\usepackage{newclude}

\usepackage[main=czech]{babel} % čeština
\usepackage{csquotes} % české uvozovky

% ------------------------------------------------------------------------------
% LITERATURA A REFERENCE
% ------------------------------------------------------------------------------

% literatura - CZ ISO 690
\usepackage[
   backend=biber
  ,style=iso-numeric
  ,sortlocale=cs_CZ
  ,sorting=nyt
  ,maxnames=3
  ,minnames=3
  ,pagetotal=true
  ,autolang=other
  ,bibencoding=UTF8
  ,spacecolon=false % mezera před dvojtečkou
]{biblatex}

\setcounter{biburllcpenalty}{7000}
\setcounter{biburlucpenalty}{8000}

\addbibresource{./literatura.bib}

% formátování seznamu zdrojů a literatury - hranaté závorky
\DeclareFieldFormat{labelnumberwidth}{[\,{#1}\,]}
\defbibenvironment{bibliography}
  {\list
     {\printtext[labelnumberwidth]{%
        %%\printfield{prefixnumber}%
        \printfield{labelprefix}%
        \printfield{labelnumber}}}
     {\setlength{\labelwidth}{\labelnumberwidth}%
      \setlength{\leftmargin}{\labelwidth}%
      \setlength{\labelsep}{\biblabelsep}%
      \addtolength{\leftmargin}{+1.5\labelsep}%
      \setlength{\itemsep}{\bibitemsep}%
      \setlength{\parsep}{\bibparsep}}%
      \renewcommand*{\makelabel}[1]{\hss##1}}
  {\endlist}
  {\item}

% ------------------------------------------------------------------------------

% přílohy
%\usepackage[titletoc]{appendix}
\usepackage{appendix}

% tabulky
\usepackage{tabularx}
\usepackage{makecell}

% poznámky pod čarou
\usepackage{footnote}

% obrázky v textu
\usepackage{wrapfig}
\usepackage{subcaption}

% matematické prostředí
\usepackage{amsmath}
\usepackage{mathtools}

% stojatá řecká písmena
\usepackage{upgreek}

% rámečky
\usepackage{framed}

% hypertext
%\usepackage{xunicode,xltxtra,url,parskip}

% předsazení odstavce
\setlength{\parindent}{1.2cm}

% zájlaví a zápatí
\usepackage{fancyhdr}

% barvy
\usepackage{xcolor}

% změna záhlaví, hypertextové odkazy bez rámečků
\usepackage{chngcntr}
\usepackage[hidelinks]{hyperref}
\hypersetup{
	colorlinks=false,
	pdfborder={0 0 0},
}

% kreslení v TikZ - grafy, diagramy, obvody
\usepackage{tikz,pgfplots}
\pgfplotsset{compat=1.18}
\usepackage[simplified]{pgf-umlcd}
\usepgfplotslibrary{polar}
\tikzset{fontscale/.style = {font=\relscale{#1}}}
\usepackage[siunitx,european]{circuitikz} % evropské značky, jednotky SI
\usetikzlibrary{intersections}

% úprava nadpisů a řádkování
\usepackage{titlesec}
\usepackage{setspace}

% ovládání software, klávesnice
\usepackage{menukeys}

% vložené strojopisy
\usepackage{verbatimbox}

\makeatletter
\newcommand\verbfilelist[2][]{%
  \setcounter{VerbboxLineNo}{0}%
  \def\verbatim@processline{%
    {\addtocounter{VerbboxLineNo}{1}%
    #1\setbox0=\hbox{#1\the\verbatim@line}%
    \hsize=\wd0 \the\verbatim@line\par}}%
  \verbatiminput{#2}
  \let\verbatim@processline\sv@verbatim@processline
}
\makeatother

% ------------------------------------------------------------------------------
% STRÁNKA A KAPITOLY
% ------------------------------------------------------------------------------

% okraje stránky
\usepackage[
  top=25mm,
  bottom=25mm,
  left=40mm,
  right=25mm,
  %includehead,
  includefoot,
  heightrounded, % underfull zarovnání
]{geometry}

%\renewcommand*\thechapter{\arabic{chapter}}

% vodící čáry kapitol
\usepackage{tocloft}
\renewcommand\cftchapdotsep{\cftdotsep}%

% tečka za číslem kapitoly v obsahu
\renewcommand{\cftchapaftersnum}{.}%

% číslování obrázků
\counterwithin{figure}{chapter}

% čísla rovnic pod kapitolou a pod podkapitolou
\counterwithin*{equation}{chapter}
\counterwithin*{equation}{section}

% čísla tabulek
\counterwithin{table}{section}

% nadpis kapitoly
\titleformat{\chapter}{\bf\raggedright\Large}{\thechapter.~}{0em}{}
\titlespacing{\chapter}{0pt}{3pt}{3pt}

% podnadpis
%\titleformat{\section}{\bf\raggedright\large}{\thechapter.\thesection.~}{0em}{}
\titlespacing{\section}{0pt}{3pt}{3pt}

% podpodnadpis
%\titleformat{\subsection}{\bf\raggedright}{\thechapter.\thesubsection.~}{0em}{}

% podpodnadpis
%\titleformat{\subsubsection}{\bf\raggedright}{\thesubsubsection.~}{0em}{}
%\titlespacing{\subsubsection}{0pt}{3pt}{3pt}

% příkazy
\newcommand{\nadpis}[1]{
    \chapter{#1}
}

\newcommand{\podnadpis}[1]{
    \section{#1}
}

\newcommand{\podpodnadpis}[1]{
    \subsection{#1}
}

% výchozí záhlaví a zápatí stránky
\lhead{}
\rhead{}
\cfoot{}

% základní styl
\fancypagestyle{StyleBase} {
	\fancyhf{}
	\lhead{}
	\rhead{}
	\cfoot{\thepage}
	\rfoot{}
}

% styl příloh
\fancypagestyle{StyleAppendix} {
	\fancyhf{}
	\lhead{}
	\rhead{}
	\cfoot{\thepage}
	\rfoot{\textit{Přílohy}}
}

% prázdná stránka
\fancypagestyle{StyleBlank} {
	\fancyhf{}
	\lhead{}
	\rhead{}
	\cfoot{}
	\rfoot{}
	\renewcommand{\headrulewidth}{0pt}
}

% ------------------------------------------------------------------------------
% VLASTNÍ PŘÍKAZY
% ------------------------------------------------------------------------------

% tečkovaná vodící čára pro podpisy
\newcommand{\dotrule}[1]{%
   \parbox[t]{#1}{\dotfill}}

% nadpis bez čísel
\newcommand{\nadpisBezCisla}[1]{
	\newpage
    \setcounter{chapter}{0}
    \setcounter{section}{0}
    \chapter*{#1}
    \addcontentsline{toc}{chapter}{#1}
}

% dvojité podtržení
\def\doubleunderline#1{\underline{\underline{#1}}}

% nota - tón
\newcommand\nota[1]{\xnote#1\relax\relax\relax}
	\def\xnote#1#2#3\relax{#1\if#2\relax\else\if b#2$\flat\if#3\relax%
	\else_{#3}\fi$\else\if###2$\sharp\if#3\relax\else_{#3}\fi$\else$_{#2}$\fi\fi\fi}

% volná, nečíslovaná stránka
\newcommand\blankpage{%
    \null
    \clearpage
    \thispagestyle{empty}%
    %\addtocounter{page}{-1}%
    \newpage
    \clearpage}

\newlength\longest

%%%%%%%%%%%%%%%%%%%%%%%%%%%%%%%%%%%%%%%%%%%%%%%%%%%%%%%%%%%%%%%%%%%%%%%%%%%%%%%%

% slovník zalamování
% slovník pro korektní zalamování slov

% en
\hyphenation{Lily-Pond}
\hyphenation{Muse-Score}


% PDF metadata
\author{\autor}
\hypersetup
{
    pdfauthor={\autor},
    pdfsubject={Absolventská písemná práce {\skolniRok}},
    pdftitle={\appTema}
}

