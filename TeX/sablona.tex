% ------------------------------------------------------------------------------
% PEVNÁ NASTAVENÍ - ŠABLONA APP KP
% ------------------------------------------------------------------------------

% formát A4, jednostranně, základní velikost písma 12 bodů
\documentclass[12pt,a4paper,oneside]{report}

% základní proklad 1.2
\renewcommand{\baselinestretch}{1.2}

% jazyk
\usepackage[main=czech]{babel} % čeština
\usepackage{csquotes} % české uvozovky

% ------------------------------------------------------------------------------
% LITERATURA A REFERENCE
% ------------------------------------------------------------------------------

% literatura - CZ ISO 690
\usepackage[
   backend=biber
  ,style=iso-numeric
  ,sortlocale=cs_CZ
  ,sorting=nyt
  ,natbib=true
  ,maxnames=3
  ,minnames=3
  ,pagetotal=true
  ,autolang=other
  ,bibencoding=UTF8
  ,url=false
  ,spacecolon=false % mezera před dvojtečkou (ISO 690:2011 2014)
]{biblatex}

\setcounter{biburllcpenalty}{7000}
\setcounter{biburlucpenalty}{8000}

% zdroj literatury
\addbibresource{./literatura.bib}

% formátování seznamu zdrojů a literatury - hranaté závorky
\DeclareFieldFormat{labelnumberwidth}{[\,{#1}\,]}
\defbibenvironment{bibliography}
  {\list
     {\printtext[labelnumberwidth]{%
        %%\printfield{prefixnumber}%
        \printfield{labelprefix}%
        \printfield{labelnumber}}}
     {\setlength{\labelwidth}{\labelnumberwidth}%
      \setlength{\leftmargin}{\labelwidth}%
      \setlength{\labelsep}{\biblabelsep}%
      \addtolength{\leftmargin}{+1.5\labelsep}%
      \setlength{\itemsep}{\bibitemsep}%
      \setlength{\parsep}{\bibparsep}}%
      \renewcommand*{\makelabel}[1]{\hss##1}}
  {\endlist}
  {\item}

% ------------------------------------------------------------------------------
% POUŽITELNÉ BALÍČKY
% ------------------------------------------------------------------------------

% výplňový text
\usepackage{lipsum}

% přílohy
\usepackage{appendix}

% tabulky
\usepackage{tabularx}
%\usepackage{makecell}

% poznámky pod čarou
\usepackage{footnote}

% obrázky v textu
\usepackage{wrapfig}
\usepackage{subcaption}

% matematické prostředí
\usepackage{amsmath}
\usepackage{mathtools}

% stojatá řecká písmena
\usepackage{upgreek}

% rámečky
\usepackage{framed}

% hypertext
\usepackage{xunicode,xltxtra,url,parskip}

% předsazení odstavce
\setlength{\parindent}{1.2cm}
\setlength{\parskip}{12pt}

% zájlaví a zápatí
\usepackage{fancyhdr}

% barvy
\usepackage{xcolor}

% změna záhlaví 
\usepackage{chngcntr}

% hypertextové odkazy bez rámečků
\usepackage[hidelinks]{hyperref}
\hypersetup{
	colorlinks=false,
	pdfborder={0 0 0},
}

% kreslení v TikZ - grafy, diagramy, obvody
\usepackage{tikz,pgfplots}
\pgfplotsset{compat=1.18}
\usepackage[simplified]{pgf-umlcd}
\usepgfplotslibrary{polar}
\tikzset{fontscale/.style = {font=\relscale{#1}}}
\usepackage[siunitx,european]{circuitikz} % evropské značky, jednotky SI
\usetikzlibrary{intersections}

% úprava nadpisů a řádkování
\usepackage{titlesec}
\usepackage{setspace}

% ovládání software, klávesnice
\usepackage{menukeys}

% vložené strojopisy
\usepackage{verbatimbox}

\makeatletter
\newcommand\verbfilelist[2][]{%
  \setcounter{VerbboxLineNo}{0}%
  \def\verbatim@processline{%
    {\addtocounter{VerbboxLineNo}{1}%
    #1\setbox0=\hbox{#1\the\verbatim@line}%
    \hsize=\wd0 \the\verbatim@line\par}}%
  \verbatiminput{#2}
  \let\verbatim@processline\sv@verbatim@processline
}
\makeatother

% vnoření PDF
\usepackage{pdfpages}

% ------------------------------------------------------------------------------
% STRÁNKA A4 - OKRAJE
% ------------------------------------------------------------------------------

% okraje stránky
\usepackage[
  top=25mm,
  bottom=25mm,
  left=40mm,
  right=25mm,
  heightrounded, 
]{geometry}

% ------------------------------------------------------------------------------
% OBSAH
% ------------------------------------------------------------------------------

% vodící čáry kapitol
\usepackage{tocloft}
\renewcommand\cftchapdotsep{\cftdotsep}%

% tečka za číslem kapitoly v obsahu
\renewcommand{\cftchapaftersnum}{.}%

% samostatný seznam příloh
\usepackage{etoolbox, titlesec}
\newcommand{\seznamPriloh}{Seznam příloh}
\newlistof{appendix}{apc}{\seznamPriloh}

\makeatletter
\AtBeginEnvironment{appendices}{%
  \clearpage
  {\protect\pagestyle{StyleAppendix}}
  \write\@auxout{%
    \string\let\string\latex@tf@toc\string\tf@toc% 
    \string\let\string\tf@toc\string\tf@apc% 
  }%
}
\AtEndEnvironment{appendices}{%
  \write\@auxout{%
    \string\let\string\tf@toc\string\latex@tf@toc% 
  }%
}
\makeatother

% ------------------------------------------------------------------------------
% STYL NADPISŮ
% ------------------------------------------------------------------------------

% nadpis kapitoly
\titleformat{\chapter}{\bf\raggedright\Large}{\thechapter.~}{0em}{}
\titlespacing*{\chapter}{0pt}{-32pt}{0pt}

% podnadpis
\titlespacing*{\section}{0pt}{0pt}{0pt}

% podpodnadpis
\titlespacing*{\subsection}{0pt}{0pt}{0pt}

% ------------------------------------------------------------------------------
% ČÍSLOVÁNÍ FIGUR
% ------------------------------------------------------------------------------

% číslování obrázků
\counterwithin{figure}{chapter}

% čísla rovnic pod kapitolou a pod podkapitolou
\counterwithin*{equation}{chapter}
\counterwithin*{equation}{section}

% čísla tabulek
\counterwithin{table}{section}

% ------------------------------------------------------------------------------
% PŘÍKAZY PRO NADPISY
% ------------------------------------------------------------------------------

% nadpis první úrovně
\newcommand{\nadpis}[1]{
    \chapter{#1}
}

% nadpis druhé úrovně
\newcommand{\podnadpis}[1]{
    \section{#1}
}

% nadpis třetí úrovně
\newcommand{\podpodnadpis}[1]{
    \subsection{#1}
}

% nadpis bez čísel
\newcommand{\nadpisBezCisla}[1]{
	\newpage
    \setcounter{chapter}{0}
    \setcounter{section}{0}
    \chapter*{#1}
    \addcontentsline{toc}{chapter}{#1}
}

% podnadpis bez čísel
\newcommand{\podnadpisBezCisla}[1]{
    \setcounter{chapter}{0}
    \setcounter{section}{0}
    \section*{#1}
    \addcontentsline{toc}{section}{#1}
}

% nadpis pro přílohy
\newcommand{\priloha}[1]{
    \chapter{#1}
    \thispagestyle{StyleAppendix}
    \pagestyle{StyleAppendix}
}

% ------------------------------------------------------------------------------
% ZÁHLAVÍ A ZÁPATÍ STRÁNEK
% ------------------------------------------------------------------------------

% výchozí záhlaví a zápatí stránky
\lhead{}
\rhead{}
\cfoot{}
\renewcommand{\headrulewidth}{0pt}
\renewcommand{\footrulewidth}{0pt}

% základní styl stránky - číslo v zápatí uprostřed
\fancypagestyle{StyleBase} {
	\fancyhf{}
	\lhead{}
	\rhead{}
	\cfoot{\thepage}
	\rfoot{}
}

% styl příloh - přidáno "Přílohy" v zápatí vpravo
\fancypagestyle{StyleAppendix} {
	\fancyhf{}
	\lhead{}
	\rhead{}
	\cfoot{\thepage}
	\rfoot{\textit{Přílohy}}
}

% ------------------------------------------------------------------------------
% JINÉ VLASTNÍ PŘÍKAZY
% ------------------------------------------------------------------------------

% tečkovaná vodící čára pro podpisy
\newcommand{\dotrule}[1]{%
   \parbox[t]{#1}{\dotfill}}

% dvojité podtržení
\def\doubleunderline#1{\underline{\underline{#1}}}

% nota - tón
\newcommand\nota[1]{\xnote#1\relax\relax\relax}
	\def\xnote#1#2#3\relax{#1\if#2\relax\else\if b#2$\flat\if#3\relax%
	\else_{#3}\fi$\else\if###2$\sharp\if#3\relax\else_{#3}\fi$\else$_{#2}$\fi\fi\fi}

% volná, nečíslovaná stránka
\newcommand\blankpage{%
    \null
    \clearpage
    \thispagestyle{empty}%
    %\addtocounter{page}{-1}%
    \newpage
    \clearpage}

\newlength\longest

% ------------------------------------------------------------------------------
% SLOVNÍK ZALAMOVÁNÍ SLOV
% ------------------------------------------------------------------------------

% slovník zalamování
% slovník pro korektní zalamování slov

% en
\hyphenation{Lily-Pond}
\hyphenation{Muse-Score}


% ------------------------------------------------------------------------------
% NASTAVENÍ VÝSLEDNÉHO PDF
% ------------------------------------------------------------------------------

% PDF metadata
\author{\autor}
\hypersetup
{
    pdfauthor={\autor},
    pdfsubject={{\typPrace}, {\skola}, {\skolniRok}},
    pdftitle={\appTema},
    pdfkeywords={{\klicovaslova}, {\klicovaslovaCizi}}
}
