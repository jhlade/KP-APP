Závěr dává odpověď cíle, otázky a~úkoly stanovené v~\textit{Úvodu práce}.
Představuje klíčovou část práce, je v~něm sdělen výsledek zkoumání. Závěr
nepředstavuje rekapitulaci předchozích částí, ale přednesení v~nich učiněných
zjištění. Součástí závěru je též tzv.~diskuse zjištěných výsledků, tj.~srovnání
svého výstupu s~jinými již publikovanými názory a~jejich vzájemné kritické
vyhodnocení. Pokud se v~práci nějakou otázku nepodařilo objasnit, je na tuto
skutečnost taktéž upozorněno. Autor by měl též naznačit charakter dalšího
zkoumání daného tématu v~budoucnosti.

\noindent
Rozsah závěru je vhodné koncipovat alespoň na~jednu celou až dvě tiskové
stránky.
