Úvod předkládá metodologickou a~koncepční charakteristiku práce. Nepovinně může
začínat autorovou motivací k~napsání dané práce, ovšem stejně jako samostatný
text by i~motivace měla být psána v~minulém čase trpného rodu a~bez výrazného
citového zabarvení.

V~úvodu musí dojít k~vymezení tématu (z~hlediska teritoriálního,
časového,\dots) a~jeho zařazení ve~stěžejních dobových reáliích. Tedy: \textit{
Co bude práce zkoumat a~co už ne. A~proč.} Dále musí být stanoveny hlavní cíle
práce - co chce práce vyřešit, objasnit, jakou otázku si klade za~cíl
zodpovědět, k~jakému konkrétnímu cíli chce dospět, a~uvedeny skutečnosti, které
naopak nebudou v~rámci daného tématu zkoumány, ať už z~důvodu rozsahu práce, či
jiných.

Úvod by měl také shrnovat, jakými metodami bude realizováno splnění vytyčených
cílů (např.~analýza, syntéza, srovnání, apod.). Nepovinně též může zahrnovat
vymezení použité terminologie, pokud bude práce specifické terminologie
potřebovat.

\podnadpisBezCisla{Stav bádání}
Stav bádání (angl. \textit{state-of-art}; spíše pouze pokud je tak rozsáhlý, že
skutečně vyžaduje samostatnou kapitolu umístěnou ihned po~Úvodu) shrnuje
a~kriticky rozebírá odbornou literaturu a~dostupné prameny vztahující se
ke~zvolenému tématu.